\documentclass[12pt]{article}

% Package for encoding and language support
\usepackage{indentfirst}

\usepackage[utf8]{inputenc}
\usepackage{amsmath}
\usepackage{amssymb}
\usepackage{amsfonts}
\usepackage{geometry}
\usepackage{graphicx}
\usepackage{fancyhdr}
\usepackage{enumitem}




% Page margins
\geometry{a4paper, margin=1in}

% Header and footer
\setlength{\headheight}{14.49998pt}
\addtolength{\topmargin}{-2.49998pt}

\pagestyle{fancy}
\fancyhf{}
\fancyhead[L]{Learning from Data - WA 1}
\fancyhead[R]{Yushan Liu}
\fancyfoot[C]{\thepage}

% Title information

\begin{document}

\begin{titlepage}
    \begin{center}
        % 插入图标
        \includegraphics[width=4cm]{tsinghua_logo.png}\\[2cm]  % 插入图标并设置下方间距
        {\Huge Writing Assignment 1} \\[2cm]
        {\large Yushan Liu  \ \  Student ID: 2024214103}\\[5cm]
        {\normalsize \today}\\[1cm]
    \end{center}
\end{titlepage}

\section*{Problem 1.1: Logistic Regression}

\subsection*{(a) Sigmoid Function}

The sigmoid function is defined as:
\[
    \sigma(z) = \frac{1}{1 + e^{-z}}
\]

We can rewrite the sigmoid function as:
\[
    \sigma(z) = \frac{e^z}{1 + e^z} = 1 - \frac{1}{1 + e^z}
\]

Then the derivative of the sigmoid function with respect to \( z \) is:
\[
    \frac{d}{dz} \sigma(z) = \frac{d}{dz} \left( 1 - \frac{1}{1 + e^z} \right) = \frac{e^z}{(1 + e^z)^2} = \frac{1}{1 + e^z} \cdot \frac{e^z}{1 + e^z} = \sigma(z) \cdot (1 - \sigma(z))
\]

Thus, the derivative of the sigmoid function with respect to \( z \) is:
\[
    \frac{d}{dz} \sigma(z) = \sigma(z) \cdot (1 - \sigma(z))
\]

\subsection*{(b) Log-Likelihood Function}

The log-likelihood function for logistic regression is given by:
\[
    \ell(\boldsymbol{\theta}) = \sum_{i=1}^{m} \left( y^{(i)} \log \sigma(\boldsymbol{\theta}^\top \boldsymbol{x}^{(i)}) + (1 - y^{(i)}) \log(1 - \sigma(\boldsymbol{\theta}^\top \boldsymbol{x}^{(i)})) \right)
\]

Since the log-likelihood is a sum over individual training examples, we can focus on the derivative for a single example \(i\):
\[
    \ell_i(\boldsymbol\theta) = y^{(i)} \log \sigma(\boldsymbol\theta^\top x^{(i)}) + (1 - y^{(i)}) \log(1 - \sigma(\boldsymbol\theta^\top x^{(i)}))
\]

The derivative of the first term \(y^{(i)} \log \sigma(\boldsymbol\theta^\top x^{(i)})\) is:


\[
    \frac{\partial}{\partial \boldsymbol\theta_j} \left( y^{(i)} \log \sigma(\boldsymbol\theta^\top x^{(i)}) \right)
    = y^{(i)} \cdot \frac{1}{\sigma(\boldsymbol\theta^\top x^{(i)})} \cdot \sigma'(\boldsymbol\theta^\top x^{(i)}) \cdot x_j^{(i)}
\]
\[
    = y^{(i)} \cdot (1 - \sigma(\boldsymbol\theta^\top x^{(i)})) \cdot x_j^{(i)}
\]

The derivative of the second term \((1 - y^{(i)}) \log(1 - \sigma(\boldsymbol\theta^\top x^{(i)}))\) is:
\[
\frac{\partial}{\partial \boldsymbol\theta_j} \left( (1 - y^{(i)}) \log(1 - \sigma(\boldsymbol\theta^\top x^{(i)})) \right)
= (1 - y^{(i)}) \cdot \frac{-1}{1 - \sigma(\boldsymbol\theta^\top x^{(i)})} \cdot (-\sigma'(\boldsymbol\theta^\top x^{(i)})) \cdot x_j^{(i)}
\]
\[
    = (1 - y^{(i)}) \cdot (- \sigma(\boldsymbol\theta^\top x^{(i)})) \cdot x_j^{(i)}
\]

Then combine the two terms and simplify, we have:
\[
    \frac{\partial}{\partial \boldsymbol\theta_j}\ell_i(\boldsymbol\theta) = (y^{(i)} - \sigma(\boldsymbol\theta^\top x^{(i)})) \cdot x_j^{(i)}
\]

Summing over all training examples \(i = 1, \ldots, m\), we obtain the desired result:
\[
\frac{\partial \ell(\boldsymbol\theta)}{\partial \boldsymbol\theta_j} = \sum_{i=1}^{m} \left( y^{(i)} - \sigma(\boldsymbol\theta^\top x^{(i)}) \right) x_j^{(i)}
\]

\section*{Problem 1.2: Ridge Regression}

\subsection*{(a) Gradient of the Ridge Regression Loss}

We are given the ridge regression loss function:
\[
J(\boldsymbol\theta) \triangleq \| \boldsymbol y - X \boldsymbol\theta \|^2 + \lambda \|\boldsymbol\theta\|^2
\]

To compute the gradient with respect to \( \theta \), we first note that the loss function can be expanded as:
\[
J(\boldsymbol\theta) = (\boldsymbol y - X\boldsymbol\theta)^\top (y - X\boldsymbol\theta) + \lambda \boldsymbol\theta^\top \boldsymbol\theta
\]

Now, differentiating \( J(\boldsymbol\theta) \) with respect to \( \boldsymbol\theta \), we get:
\[
\nabla_{\boldsymbol\theta} J(\boldsymbol\theta) = -2X^\top(\boldsymbol y - X \boldsymbol\theta) + 2\lambda\boldsymbol\theta
\]

Thus, the gradient is:
\[
\nabla_{\boldsymbol\theta} J(\boldsymbol\theta) = 2X^\top X \boldsymbol\theta - 2X^\top \boldsymbol y + 2\lambda \boldsymbol\theta
\]

\subsection*{(b) Gradient Descent Update Rule}

Using the gradient computed above, the update rule for gradient descent is:
\[
\theta_{t+1} \colon= \theta_{t} + \alpha \nabla_\theta J(\theta)
\]

Substituting the gradient, we have:
\[
    \theta_{t+1} \colon= \theta_{t} + \alpha \left( 2X^\top X \theta_t - 2X^\top \boldsymbol y + 2\lambda \theta_t \right)
\]

\subsection*{(c) Optimal Solution Using the Normal Equation}

The optimal parameter \( \boldsymbol\theta^* \) can be derived by setting the gradient to zero:
\[
2X^\top X \boldsymbol\theta^* - 2X^\top \boldsymbol y + 2\lambda \boldsymbol\theta^* = 0
\]

Simplifying, we get the normal equation:
\[
(X^\top X +\lambda \boldsymbol I)\boldsymbol\theta^* = X^\top \boldsymbol y
\]

Solving for \( \theta^* \), we obtain:
\[
    \boldsymbol\theta^* = (X^\top X + \lambda \boldsymbol I)^{-1} X^\top \boldsymbol y
\]

\newpage

\section*{Problem 1.3: Poisson Distribution and Generalized Linear Model (GLM)}

\subsection*{(a) Exponential Family Form of the Poisson Distribution}

The probability mass function of the Poisson distribution is given by:
\[
p(y \mid \lambda) = \frac{\lambda^y e^{-\lambda}}{y!}
\]

We have knowed that a class of distributions is in the exponential family if its density can be
written in the canonical form:
\[
p(y \mid \eta) = h(y) \exp\left( \eta T(y) - a(\eta) \right)
\]

Rewriting in the exponential family form for the Poisson distribution, we have:
\[
p(y \mid \eta) = \frac{1}{y!} \exp\left( \eta y - e^{\eta} \right)
\]

We can see that:

\begin{itemize}
    \item \( \eta = \log(\lambda) \) is the natural parameter.
    \item \( T(y) = y \) is the sufficient statistic.
    \item \( a(\eta) = e^{\eta} \) is the log-partition function.
    \item \( b(y) = \frac{1}{y!} \) normalizes the distribution.
\end{itemize}

\subsection*{(b) GLM for Poisson Regression}
From solving \textbf{(a)}, we know that:  

\begin{itemize}
    \item \( \eta = \log(\lambda) \) is the natural parameter.
    \item \( T(y) = y \) is the sufficient statistic.
\end{itemize}

By deriving hypothesis function from the exponential family form, we have:
\[
h_\theta(x) = E[T(y)|x; \theta]= \lambda = \eta
\]

To adopt linear model \(\eta = \theta^T x\), we have:
\[
    \log (\lambda) = \eta = \theta^T X
\]
\[
    h_\theta(x) = e^{\theta^\top x}
\]

Thus, we can conclude that:
\begin{itemize}
    \item \textbf{Hypothesis function}: \( h_\theta(x) = e^{\theta^\top x} \), where \( \theta^\top x \) is the linear combination of the input features \( x \).
    \item \textbf{Canonical link function}: \( g(\lambda) = \log(\lambda) \), which relates the rate parameter \( \lambda \) to the natural parameter \( \eta = \theta^\top x \).
    \item \textbf{Inverse canonical link function}: \( g^{-1}(\eta) = e^{\eta} \), which transforms the natural parameter \( \eta \) back into the rate parameter \( \lambda \).
\end{itemize}

\section*{Problem 1.4: Softmax Regression}

The Softmax model's log-likelihood function is given by:
\[
\ell(\boldsymbol\Theta) \triangleq \sum_{i=1}^m \log p(y^{(i)} | x^{(i)}; \boldsymbol\Theta) = \sum_{i=1}^m \sum_{l=1}^K \mathbf{1}\{ y^{(i)} = l \} \log \frac{e^{\boldsymbol\theta_l^\top x^{(i)}}}{\sum_{j=1}^K e^{\boldsymbol\theta_j^\top x^{(i)}}}
\]

We can express the log-likelihood function in terms of the indicator function and the softmax probabilities:
\[
p(y^{(i)} = l \mid x^{(i)}; \boldsymbol\Theta) = \frac{e^{\boldsymbol\theta_l^\top x^{(i)}}}{\sum_{j=1}^K e^{\boldsymbol\theta_j^\top x^{(i)}}}
\]

The full log-likelihood can be written as:
\[
\ell(\Theta) = \sum_{i=1}^m \log \left( \frac{e^{\theta_{y^{(i)}}^\top x^{(i)}}}{\sum_{j=1}^K e^{\theta_j^\top x^{(i)}}} \right)= \sum_{i=1}^m \left( \theta_{y^{(i)}}^\top x^{(i)} - \log \left( \sum_{j=1}^K e^{\theta_j^\top x^{(i)}} \right) \right)
\]

What we need to calculate is:
\[
    \nabla_{\boldsymbol\theta_l}\ell(\boldsymbol{\Theta}) =  \sum_{i=1}^m \frac{\partial}{\partial \theta_l} \left( \theta_{y^{(i)}}^\top x^{(i)} - \log \left( \sum_{j=1}^K e^{\theta_j^\top x^{(i)}} \right) \right)
\]
We can take the derivative term-by-term:

1. Derivative of the first term\( \theta_{y^{(i)}}^\top x^{(i)} \)

\begin{itemize}
    \item For \( l = y^{(i)} \), \( \frac{\partial}{\partial \theta_l} \theta_{y^{(i)}}^\top x^{(i)} = x^{(i)} \).
    \item For \( l \neq y^{(i)} \), \( \frac{\partial}{\partial \theta_l} \theta_{y^{(i)}}^\top x^{(i)} = 0 \).
\end{itemize}

2. Derivative of the second term \( - \log \left( \sum_{j=1}^K e^{\theta_j^\top x^{(i)}} \right) \)
\[
    \frac{\partial}{\partial \theta_l} \left( - \log \left( \sum_{j=1}^K e^{\theta_j^\top x^{(i)}} \right) \right) = - \frac{1}{\sum_{j=1}^K e^{\theta_j^\top x^{(i)}}} \cdot \frac{\partial}{\partial \theta_l} \left( \sum_{j=1}^K e^{\theta_j^\top x^{(i)}} \right)
\]
\[
    = - \frac{1}{\sum_{j=1}^K e^{\theta_j^\top x^{(i)}}} \cdot  e^{\theta_l^\top x^{(i)}} \cdot x^{(i)} = - P(y = l \mid x^{(i)}; \Theta) \cdot x^{(i)}
\]

So for each class \(l\), the gradient of the log-likelihood with respect to \(\theta_l\) is:
\[
    \nabla_{\boldsymbol\theta_l}\ell(\boldsymbol{\Theta}) = \sum_{i=1}^m \left( \mathbf{1}\{ y^{(i)} = l \} - P(y = l \mid x^{(i)}; \Theta) \right) x^{(i)}
\]

Where:
\begin{itemize}
    \item \( \mathbf{1}\{ y^{(i)} = l \} \) is 1 if the \( i \)-th example belongs to class \( l \), and 0 otherwise.
    \item \( P(y = l \mid x^{(i)}; \Theta) = \frac{e^{\theta_l^\top x^{(i)}}}{\sum_{j=1}^K e^{\theta_j^\top x^{(i)}}} \) is the softmax probability.
\end{itemize}

\newpage

\section*{Problem 1.5: Maximun Likelihood Estimation}

\subsection*{(a) the Expression of Conditional Distribution}

The conditional distribution of \( y \) given \(\boldsymbol x \) is the distribution of \( y - \boldsymbol\theta^\top \boldsymbol x \), which is simply the distribution of the error term \( \epsilon \). Hence, the conditional distribution of \( y \) given \( x \) is:
\[
    P_{Y|X}(y|\boldsymbol x; \boldsymbol\theta) = \frac{1}{2\tau} \exp\left( -\frac{|y - \boldsymbol\theta^\top \boldsymbol x|}{\tau} \right)
\]

\subsection*{(b) the Log-Likelihood Function}

Given the conditional probability \( P_{Y|X}(y|\boldsymbol x; \boldsymbol\theta) \), 
the log-likelihood for \( m \) samples \( \{(x^{(i)}, y^{(i)})\}_{i=1}^m \) can be written as:
\[
    \ell(\boldsymbol\theta) = \sum_{i=1}^m \log P_{Y|X}(y^{(i)} | x^{(i)}; \boldsymbol\theta) =  \sum_{i=1}^m \log \left( \frac{1}{2\tau} \exp\left( -\frac{|y^{(i)} - \boldsymbol\theta^\top x^{(i)}|}{\tau} \right) \right)
\]
\[
    =  \sum_{i=1}^m \left( \log \left( \frac{1}{2\tau} \right) - \frac{|y^{(i)} - \boldsymbol\theta^\top x^{(i)}|}{\tau} \right) 
\]
\[
    =  -m \log(2\tau) - \frac{1}{\tau} \sum_{i=1}^m |y^{(i)} - \boldsymbol\theta^\top x^{(i)}|
\]

\subsection*{(c) the Geometric Interpretation of LAD}

In ordinary least squares (OLS) regression, we minimize the sum of the squared distances between the predicted and actual values, effectively finding a line that minimizes the \(\textbf{squared Euclidean distance}\) between the points and the regression line. This gives the usual \( \ell_2 \)-norm, which is sensitive to outliers because outliers have a disproportionately large influence due to the squaring of distances.

In least absolute deviation(LAD) regression, we minimize the sum of the absolute deviations \( |y^{(i)} - \theta^\top x^{(i)}| \), which corresponds to the \( \ell_1 \)-norm.

The geometric interpretation of LAD is that instead of minimizing the Euclidean distance, we are minimizing the \(\textbf{Manhattan distance}\), or the \(\textbf{vertical distances}\) between the data points and the regression line. 

This results in a model that is more \(\textbf{robust to outliers}\) because outliers have a linear influence on the objective function, as opposed to a quadratic influence in OLS.


\newpage
\bibliographystyle{plain}  % 使用plain样式来展示参考文献
\bibliography{references}   % references.bib 是存放参考文献条目的文件
\end{document}
